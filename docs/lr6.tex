\documentclass[12pt,a4paper]{extarticle}
\usepackage{setspace}
\usepackage{indentfirst}
\usepackage{fancyhdr}
\usepackage{lastpage}
\usepackage{fontspec}
\usepackage{geometry}

% Шрифт
\setmainfont{Times New Roman}[
    Path = ./fonts/,
    UprightFont = times.ttf,
    BoldFont = timesbd.ttf,
    ItalicFont = timesi.ttf,
    BoldItalicFont = timesbi.ttf
]

% Поля документа
\geometry{left=2cm}% левое поле
\geometry{right=1.5cm}% правое поле
\geometry{top=1cm}% верхнее поле
\geometry{bottom=2cm}% нижнее поле

\newcommand{\imgh}[3] {
    \begin{figure}[h]
    \center{\includegraphics[width=#1]{#2}}
    \caption{#3}
    \label{ris:#2}
    \end{figure}
}

% Полуторный межстрочный
\onehalfspacing

% Красная строка
\setlength{\parindent}{1.25cm}

\newcommand{\maketitlepage}[1]{
\begin{titlepage}
\thispagestyle{empty}

% Верхняя часть - по центру
\begin{center}
\LARGE Уральский федеральный университет
\end{center}

% Пропускаем пространство до середины
\vspace*{\stretch{2}}

% Средняя часть - по центру
\begin{center}
\bfseries\Huge Отчет по \\
\bfseries\Huge Лабораторной работе №#1
\end{center}

% Пропускаем немного
\vspace*{\stretch{3}}

% Информация о студенте - выравнивание по левому краю
\begin{flushleft}
\large
\textbf{Студент:} Воробьев Михаил Сергеевич, \\
\textbf{Группа:} РИЗ-150916у
\end{flushleft}

% Пропускаем до низа
\vfill

% Нижняя часть - по центру
\begin{center}
\large Екатеринбург, \\
\large 2026 год
\end{center}

\end{titlepage}
}
\usepackage{listings}
\usepackage{xcolor}

\lstset{
    basicstyle=\ttfamily\small,
    keywordstyle=\color{blue},
    commentstyle=\color{gray},
    stringstyle=\color{red},
    frame=single,
    breaklines=true,
    showstringspaces=false
}

\begin{document}

\maketitlepage{6}

% Оглавление
\tableofcontents
\newpage

% Разделы
\section{Цель работы}
\normalfont Освоить работу со статическими полями и методами, перегрузкой методов, аргументами переменной длины (varargs) и операциями над массивами в Java.

\section{Описание задачи}

Реализовать 10 программ в пакете lr6, демонстрирующих различные аспекты работы с методами и массивами.

\begin{enumerate}
    \item Класс с перегруженным методом для присваивания значений полям
    \item Класс со статическим полем-счётчиком
    \item Статические методы max, min, avg с varargs
    \item Статический метод для двойного факториала
    \item Статический метод для суммы квадратов
    \item Статический метод для получения подмассива
    \item Статический метод преобразования символов в коды
    \item Статический метод для среднего значения массива
    \item Статический метод для обращения массива
    \item Статический метод для нахождения min и max с varargs
\end{enumerate}

\section{Ход выполнения}

\subsection{Задание 1: Перегруженный метод присваивания}

Класс SymbolText содержит символьное и текстовое поле. Метод setValue перегружен для работы с символом, строкой и символьным массивом.

\begin{lstlisting}[language=Java]
package lr6;

public class Task1 {
    static class SymbolText {
        public char symbol;
        public String text;

        public void setValue(char c) {
            symbol = c;
        }

        public void setValue(String s) {
            text = s;
        }

        public void setValue(char[] arr) {
            if (arr.length == 1) {
                symbol = arr[0];
            } else {
                text = new String(arr);
            }
        }

        public void print() {
            System.out.println("Symbol: " + symbol + ", Text: " + text);
        }
    }

    public static void main(String[] args) {
        SymbolText obj = new SymbolText();
        obj.setValue('A');
        obj.setValue("Hello");
        obj.print();
        obj.setValue(new char[]{'W', 'o', 'r', 'l', 'd'});
        obj.print();
    }
}
\end{lstlisting}

\subsection{Задание 2: Статический счётчик}

Класс Counter содержит закрытое статическое поле, которое увеличивается при каждом вызове метода showAndIncrement.

\begin{lstlisting}[language=Java]
package lr6;

public class Task2 {
    static class Counter {
        private static int count = 0;

        public static void showAndIncrement() {
            System.out.println("Current value: " + count);
            count++;
        }
    }

    public static void main(String[] args) {
        Counter.showAndIncrement();
        Counter.showAndIncrement();
        Counter.showAndIncrement();
    }
}
\end{lstlisting}

\subsection{Задание 3: Методы max, min, avg}

Статические методы принимают произвольное количество аргументов (varargs) и вычисляют максимум, минимум и среднее значение.

\begin{lstlisting}[language=Java]
package lr6;

public class Task3 {
    public static int max(int... numbers) {
        int result = numbers[0];
        for (int n : numbers) {
            if (n > result) result = n;
        }
        return result;
    }

    public static int min(int... numbers) {
        int result = numbers[0];
        for (int n : numbers) {
            if (n < result) result = n;
        }
        return result;
    }

    public static double avg(int... numbers) {
        int sum = 0;
        for (int n : numbers) {
            sum += n;
        }
        return (double) sum / numbers.length;
    }

    public static void main(String[] args) {
        System.out.println("max(3,7,2,9,4) = " + max(3, 7, 2, 9, 4));
        System.out.println("min(3,7,2,9,4) = " + min(3, 7, 2, 9, 4));
        System.out.println("avg(3,7,2,9,4) = " + avg(3, 7, 2, 9, 4));
    }
}
\end{lstlisting}

\subsection{Задание 4: Двойной факториал}

Метод вычисляет двойной факториал n!! — произведение чисел через одно от n до 1 или 2.

\begin{lstlisting}[language=Java]
package lr6;

public class Task4 {
    public static int doubleFactorial(int n) {
        int result = 1;
        for (int i = n; i >= 1; i -= 2) {
            result *= i;
        }
        return result;
    }

    public static void main(String[] args) {
        System.out.println("6!! = " + doubleFactorial(6));
        System.out.println("5!! = " + doubleFactorial(5));
    }
}
\end{lstlisting}

\subsection{Задание 5: Сумма квадратов}

Метод вычисляет сумму $1^2 + 2^2 + ... + n^2$. Для проверки используется формула $n(n+1)(2n+1)/6$.

\begin{lstlisting}[language=Java]
package lr6;

public class Task5 {
    public static int sumOfSquares(int n) {
        int sum = 0;
        for (int i = 1; i <= n; i++) {
            sum += i * i;
        }
        return sum;
    }

    public static int sumOfSquaresFormula(int n) {
        return n * (n + 1) * (2 * n + 1) / 6;
    }

    public static void main(String[] args) {
        int n = 10;
        System.out.println("Sum (loop): " + sumOfSquares(n));
        System.out.println("Sum (formula): " + sumOfSquaresFormula(n));
    }
}
\end{lstlisting}

\subsection{Задание 6: Подмассив из начальных элементов}

Метод возвращает новый массив из первых count элементов исходного массива. Если count больше длины массива, возвращается копия.

\begin{lstlisting}[language=Java]
package lr6;

import java.util.Arrays;

public class Task6 {
    public static int[] takeFirst(int[] arr, int count) {
        if (count >= arr.length) {
            return Arrays.copyOf(arr, arr.length);
        }
        return Arrays.copyOf(arr, count);
    }

    public static void main(String[] args) {
        int[] original = {1, 2, 3, 4, 5, 6, 7, 8, 9, 10};
        System.out.println("Original: " + Arrays.toString(original));
        System.out.println("First 3: " + Arrays.toString(takeFirst(original, 3)));
        System.out.println("First 15: " + Arrays.toString(takeFirst(original, 15)));
    }
}
\end{lstlisting}

\subsection{Задание 7: Символы в коды}

Метод преобразует массив символов в массив их числовых кодов.

\begin{lstlisting}[language=Java]
package lr6;

import java.util.Arrays;

public class Task7 {
    public static int[] charsToCodes(char[] chars) {
        int[] codes = new int[chars.length];
        for (int i = 0; i < chars.length; i++) {
            codes[i] = (int) chars[i];
        }
        return codes;
    }

    public static void main(String[] args) {
        char[] symbols = {'A', 'B', 'C', 'Z'};
        System.out.println("Symbols: " + Arrays.toString(symbols));
        System.out.println("Codes: " + Arrays.toString(charsToCodes(symbols)));
    }
}
\end{lstlisting}

\subsection{Задание 8: Среднее значение массива}

Метод вычисляет среднее арифметическое элементов целочисленного массива.

\begin{lstlisting}[language=Java]
package lr6;

import java.util.Arrays;

public class Task8 {
    public static double average(int[] arr) {
        int sum = 0;
        for (int n : arr) {
            sum += n;
        }
        return (double) sum / arr.length;
    }

    public static void main(String[] args) {
        int[] arr = {1, 2, 3, 4, 5};
        System.out.println("Array: " + Arrays.toString(arr));
        System.out.println("Average: " + average(arr));
    }
}
\end{lstlisting}

\subsection{Задание 9: Обращение массива}

Метод меняет элементы массива попарно: первый с последним, второй с предпоследним и т.д.

\begin{lstlisting}[language=Java]
package lr6;

import java.util.Arrays;

public class Task9 {
    public static void reverseSwap(char[] arr) {
        int left = 0;
        int right = arr.length - 1;
        while (left < right) {
            char temp = arr[left];
            arr[left] = arr[right];
            arr[right] = temp;
            left++;
            right--;
        }
    }

    public static void main(String[] args) {
        char[] arr = {'A', 'B', 'C', 'D', 'E'};
        System.out.println("Before: " + Arrays.toString(arr));
        reverseSwap(arr);
        System.out.println("After: " + Arrays.toString(arr));
    }
}
\end{lstlisting}

\subsection{Задание 10: Минимум и максимум}

Метод принимает произвольное количество аргументов и возвращает массив из двух элементов: минимального и максимального значений.

\begin{lstlisting}[language=Java]
package lr6;

import java.util.Arrays;

public class Task10 {
    public static int[] minMax(int... numbers) {
        int min = numbers[0];
        int max = numbers[0];
        for (int n : numbers) {
            if (n < min) min = n;
            if (n > max) max = n;
        }
        return new int[]{min, max};
    }

    public static void main(String[] args) {
        int[] result = minMax(5, 2, 8, 1, 9, 3);
        System.out.println("minMax(5,2,8,1,9,3) = " + Arrays.toString(result));
    }
}
\end{lstlisting}

\section{Ссылка на GitHub-репозиторий}

Исходный код лабораторной работы доступен по ссылке:

\url{https://github.com/viklover/urfu}

Файлы классов расположены в директории \texttt{src/lr6/}.

\section{Вывод}

В ходе лабораторной работы были реализованы программы, демонстрирующие работу со статическими полями и методами, перегрузкой методов, аргументами переменной длины (varargs) и различными операциями над массивами. Цель работы достигнута.

\end{document}
