\documentclass[12pt,a4paper]{extarticle}
\usepackage{setspace}
\usepackage{indentfirst}
\usepackage{fancyhdr}
\usepackage{lastpage}
\usepackage{fontspec}
\usepackage{geometry}

% Шрифт
\setmainfont{Times New Roman}[
    Path = ./fonts/,
    UprightFont = times.ttf,
    BoldFont = timesbd.ttf,
    ItalicFont = timesi.ttf,
    BoldItalicFont = timesbi.ttf
]

% Поля документа
\geometry{left=2cm}% левое поле
\geometry{right=1.5cm}% правое поле
\geometry{top=1cm}% верхнее поле
\geometry{bottom=2cm}% нижнее поле

\newcommand{\imgh}[3] {
    \begin{figure}[h]
    \center{\includegraphics[width=#1]{#2}}
    \caption{#3}
    \label{ris:#2}
    \end{figure}
}

% Полуторный межстрочный
\onehalfspacing

% Красная строка
\setlength{\parindent}{1.25cm}

\newcommand{\maketitlepage}[1]{
\begin{titlepage}
\thispagestyle{empty}

% Верхняя часть - по центру
\begin{center}
\LARGE Уральский федеральный университет
\end{center}

% Пропускаем пространство до середины
\vspace*{\stretch{2}}

% Средняя часть - по центру
\begin{center}
\bfseries\Huge Отчет по \\
\bfseries\Huge Лабораторной работе №#1
\end{center}

% Пропускаем немного
\vspace*{\stretch{3}}

% Информация о студенте - выравнивание по левому краю
\begin{flushleft}
\large
\textbf{Студент:} Воробьев Михаил Сергеевич, \\
\textbf{Группа:} РИЗ-150916у
\end{flushleft}

% Пропускаем до низа
\vfill

% Нижняя часть - по центру
\begin{center}
\large Екатеринбург, \\
\large 2026 год
\end{center}

\end{titlepage}
}
\usepackage{listings}
\usepackage{xcolor}

\lstset{
    basicstyle=\ttfamily\small,
    keywordstyle=\color{blue},
    commentstyle=\color{gray},
    stringstyle=\color{red},
    frame=single,
    breaklines=true,
    showstringspaces=false
}

\begin{document}

\maketitlepage{3}

% Оглавление
\tableofcontents
\newpage

% Разделы
\section{Цель работы}
\normalfont Освоить работу с условными операторами (if-else, switch), циклами (for, while) и одномерными массивами в языке Java.

\section{Описание задачи}

Реализовать 10 заданий в пакете lr3. Для некоторых заданий требуется несколько версий с использованием разных конструкций языка.

\begin{enumerate}
    \item Определение дня недели по номеру (switch)
    \item Определение номера дня по названию (две версии: if-else и switch)
    \item Вывод чисел Фибоначчи (версии с разными циклами)
    \item Вывод чисел от меньшего к большему (версии с разными циклами)
    \item Сумма чисел по условию (версии с разными циклами)
    \item Массив чисел вида 2, 7, 12, 17...
    \item Символьный массив с буквами через одну
    \item Массив прописных согласных английского алфавита
    \item Поиск минимума и его индексов в массиве
    \item Сортировка массива по убыванию
\end{enumerate}

\section{Ход выполнения}

\subsection{Задание 1: День недели по номеру}

Программа определяет название дня недели по введённому номеру от 1 до 7. Используется оператор switch с обработкой некорректного ввода в блоке default.

\begin{lstlisting}[language=Java]
package lr3;

import java.util.Scanner;

public class Task1 {
    public static void main(String[] args) {
        Scanner in = new Scanner(System.in);
        System.out.println("Введите число от 1 до 7:");
        int day = in.nextInt();
        switch (day) {
            case 1: System.out.println("Понедельник"); break;
            case 2: System.out.println("Вторник"); break;
            case 3: System.out.println("Среда"); break;
            case 4: System.out.println("Четверг"); break;
            case 5: System.out.println("Пятница"); break;
            case 6: System.out.println("Суббота"); break;
            case 7: System.out.println("Воскресенье"); break;
            default:
                System.out.println("Ошибка: число должно быть от 1 до 7");
        }
        in.close();
    }
}
\end{lstlisting}

\subsection{Задание 2: Номер дня по названию}

\subsubsection{Версия с if-else}

Программа определяет порядковый номер дня недели по его названию. Используется цепочка вложенных условий if-else.

\begin{lstlisting}[language=Java]
package lr3;

import java.util.Scanner;

public class Task2IfElse {
    public static void main(String[] args) {
        Scanner in = new Scanner(System.in);
        System.out.println("Введите название дня недели:");
        String day = in.nextLine().toLowerCase();
        int number;
        if (day.equals("понедельник")) {
            number = 1;
        } else if (day.equals("вторник")) {
            number = 2;
        } else if (day.equals("среда")) {
            number = 3;
        } else if (day.equals("четверг")) {
            number = 4;
        } else if (day.equals("пятница")) {
            number = 5;
        } else if (day.equals("суббота")) {
            number = 6;
        } else if (day.equals("воскресенье")) {
            number = 7;
        } else {
            System.out.println("Ошибка: некорректное название");
            in.close();
            return;
        }
        System.out.println("Порядковый номер: " + number);
        in.close();
    }
}
\end{lstlisting}

\subsubsection{Версия с switch}

Аналогичная логика, реализованная через оператор switch по строке.

\begin{lstlisting}[language=Java]
package lr3;

import java.util.Scanner;

public class Task2Switch {
    public static void main(String[] args) {
        Scanner in = new Scanner(System.in);
        System.out.println("Введите название дня недели:");
        String day = in.nextLine().toLowerCase();
        int number;
        switch (day) {
            case "понедельник": number = 1; break;
            case "вторник": number = 2; break;
            case "среда": number = 3; break;
            case "четверг": number = 4; break;
            case "пятница": number = 5; break;
            case "суббота": number = 6; break;
            case "воскресенье": number = 7; break;
            default:
                System.out.println("Ошибка: некорректное название");
                in.close();
                return;
        }
        System.out.println("Порядковый номер: " + number);
        in.close();
    }
}
\end{lstlisting}

\subsection{Задание 3: Числа Фибоначчи}

Программа выводит последовательность чисел Фибоначчи, где первые два числа равны 1.

\subsubsection{Версия с циклом for}

\begin{lstlisting}[language=Java]
package lr3;

import java.util.Scanner;

public class Task3For {
    public static void main(String[] args) {
        Scanner in = new Scanner(System.in);
        System.out.println("Введите количество чисел Фибоначчи:");
        int n = in.nextInt();
        if (n <= 0) {
            System.out.println("Количество должно быть положительным");
            in.close();
            return;
        }
        int a = 1, b = 1;
        for (int i = 0; i < n; i++) {
            System.out.print(a + " ");
            int temp = a + b;
            a = b;
            b = temp;
        }
        System.out.println();
        in.close();
    }
}
\end{lstlisting}

\subsubsection{Версия с циклом while}

\begin{lstlisting}[language=Java]
package lr3;

import java.util.Scanner;

public class Task3While {
    public static void main(String[] args) {
        Scanner in = new Scanner(System.in);
        System.out.println("Введите количество чисел Фибоначчи:");
        int n = in.nextInt();
        if (n <= 0) {
            System.out.println("Количество должно быть положительным");
            in.close();
            return;
        }
        int a = 1, b = 1;
        int i = 0;
        while (i < n) {
            System.out.print(a + " ");
            int temp = a + b;
            a = b;
            b = temp;
            i++;
        }
        System.out.println();
        in.close();
    }
}
\end{lstlisting}

\subsection{Задание 4: Числа от меньшего к большему}

Программа выводит все целые числа между двумя введёнными значениями.

\subsubsection{Версия с циклом for}

\begin{lstlisting}[language=Java]
package lr3;

import java.util.Scanner;

public class Task4For {
    public static void main(String[] args) {
        Scanner in = new Scanner(System.in);
        System.out.println("Введите первое число:");
        int a = in.nextInt();
        System.out.println("Введите второе число:");
        int b = in.nextInt();
        int min = Math.min(a, b);
        int max = Math.max(a, b);
        for (int i = min; i <= max; i++) {
            System.out.print(i + " ");
        }
        System.out.println();
        in.close();
    }
}
\end{lstlisting}

\subsubsection{Версия с циклом while}

\begin{lstlisting}[language=Java]
package lr3;

import java.util.Scanner;

public class Task4While {
    public static void main(String[] args) {
        Scanner in = new Scanner(System.in);
        System.out.println("Введите первое число:");
        int a = in.nextInt();
        System.out.println("Введите второе число:");
        int b = in.nextInt();
        int min = Math.min(a, b);
        int max = Math.max(a, b);
        int i = min;
        while (i <= max) {
            System.out.print(i + " ");
            i++;
        }
        System.out.println();
        in.close();
    }
}
\end{lstlisting}

\subsection{Задание 5: Сумма чисел по условию}

Программа находит сумму чисел от 1 до n, которые при делении на 5 дают остаток 2 или при делении на 3 дают остаток 1.

\subsubsection{Версия с циклом for}

\begin{lstlisting}[language=Java]
package lr3;

import java.util.Scanner;

public class Task5For {
    public static void main(String[] args) {
        Scanner in = new Scanner(System.in);
        System.out.println("Введите количество чисел:");
        int n = in.nextInt();
        int sum = 0;
        System.out.println("Числа, удовлетворяющие условию:");
        for (int i = 1; i <= n; i++) {
            if (i % 5 == 2 || i % 3 == 1) {
                System.out.print(i + " ");
                sum += i;
            }
        }
        System.out.println();
        System.out.println("Сумма: " + sum);
        in.close();
    }
}
\end{lstlisting}

\subsubsection{Версия с циклом while}

\begin{lstlisting}[language=Java]
package lr3;

import java.util.Scanner;

public class Task5While {
    public static void main(String[] args) {
        Scanner in = new Scanner(System.in);
        System.out.println("Введите количество чисел:");
        int n = in.nextInt();
        int sum = 0;
        int i = 1;
        System.out.println("Числа, удовлетворяющие условию:");
        while (i <= n) {
            if (i % 5 == 2 || i % 3 == 1) {
                System.out.print(i + " ");
                sum += i;
            }
            i++;
        }
        System.out.println();
        System.out.println("Сумма: " + sum);
        in.close();
    }
}
\end{lstlisting}

\subsection{Задание 6: Массив чисел 2, 7, 12, 17...}

Программа создаёт массив чисел, дающих остаток 2 при делении на 5. Формула элемента: $a_i = 2 + 5i$.

\begin{lstlisting}[language=Java]
package lr3;

import java.util.Scanner;

public class Task6 {
    public static void main(String[] args) {
        Scanner in = new Scanner(System.in);
        System.out.println("Введите размер массива:");
        int n = in.nextInt();
        if (n <= 0) {
            System.out.println("Ошибка: размер должен быть положительным");
            in.close();
            return;
        }
        int[] arr = new int[n];
        for (int i = 0; i < n; i++) {
            arr[i] = 2 + i * 5;
        }
        System.out.println("Массив:");
        for (int i = 0; i < n; i++) {
            System.out.print(arr[i] + " ");
        }
        System.out.println();
        in.close();
    }
}
\end{lstlisting}

\subsection{Задание 7: Символьный массив через одну букву}

Программа создаёт массив из 10 символов, заполняя его буквами русского алфавита через одну, начиная с 'а'. Массив выводится в прямом и обратном порядке.

\begin{lstlisting}[language=Java]
package lr3;

public class Task7 {
    public static void main(String[] args) {
        int size = 10;
        char[] arr = new char[size];
        char c = 'а';
        for (int i = 0; i < size; i++) {
            arr[i] = c;
            c += 2;
        }
        System.out.println("Массив в прямом порядке:");
        for (int i = 0; i < size; i++) {
            System.out.print(arr[i] + " ");
        }
        System.out.println();
        System.out.println("Массив в обратном порядке:");
        for (int i = size - 1; i >= 0; i--) {
            System.out.print(arr[i] + " ");
        }
        System.out.println();
    }
}
\end{lstlisting}

\subsection{Задание 8: Прописные согласные}

Программа создаёт массив из 10 прописных согласных букв английского алфавита, пропуская гласные (A, E, I, O, U, Y).

\begin{lstlisting}[language=Java]
package lr3;

public class Task8 {
    public static void main(String[] args) {
        char[] arr = new char[10];
        String vowels = "AEIOUY";
        int index = 0;
        char c = 'A';
        while (index < 10 && c <= 'Z') {
            if (vowels.indexOf(c) == -1) {
                arr[index] = c;
                index++;
            }
            c++;
        }
        System.out.println("Прописные согласные английского алфавита:");
        for (int i = 0; i < 10; i++) {
            System.out.print(arr[i] + " ");
        }
        System.out.println();
    }
}
\end{lstlisting}

\subsection{Задание 9: Минимум и его индексы}

Программа создаёт массив случайных чисел, находит минимальное значение и выводит все индексы, содержащие это значение.

\begin{lstlisting}[language=Java]
package lr3;

import java.util.Random;

public class Task9 {
    public static void main(String[] args) {
        int size = 10;
        int[] arr = new int[size];
        Random rand = new Random();
        System.out.println("Массив:");
        for (int i = 0; i < size; i++) {
            arr[i] = rand.nextInt(20);
            System.out.print(arr[i] + " ");
        }
        System.out.println();
        int min = arr[0];
        for (int i = 1; i < size; i++) {
            if (arr[i] < min) {
                min = arr[i];
            }
        }
        System.out.println("Минимальное значение: " + min);
        System.out.print("Индексы минимального элемента: ");
        for (int i = 0; i < size; i++) {
            if (arr[i] == min) {
                System.out.print(i + " ");
            }
        }
        System.out.println();
    }
}
\end{lstlisting}

\subsection{Задание 10: Сортировка по убыванию}

Программа создаёт массив случайных чисел и сортирует его по убыванию методом пузырька.

\begin{lstlisting}[language=Java]
package lr3;

import java.util.Random;

public class Task10 {
    public static void main(String[] args) {
        int size = 10;
        int[] arr = new int[size];
        Random rand = new Random();
        System.out.println("Исходный массив:");
        for (int i = 0; i < size; i++) {
            arr[i] = rand.nextInt(100);
            System.out.print(arr[i] + " ");
        }
        System.out.println();
        for (int i = 0; i < size - 1; i++) {
            for (int j = 0; j < size - 1 - i; j++) {
                if (arr[j] < arr[j + 1]) {
                    int temp = arr[j];
                    arr[j] = arr[j + 1];
                    arr[j + 1] = temp;
                }
            }
        }
        System.out.println("Отсортированный по убыванию:");
        for (int i = 0; i < size; i++) {
            System.out.print(arr[i] + " ");
        }
        System.out.println();
    }
}
\end{lstlisting}

\section{Ссылка на GitHub-репозиторий}

Исходный код лабораторной работы доступен по ссылке:

\url{https://github.com/viklover/urfu}

Файлы классов расположены в директории \texttt{src/lr3/}.

\section{Вывод}

В ходе лабораторной работы были реализованы программы, демонстрирующие работу с условными операторами if-else и switch, циклами for и while, а также одномерными массивами. Освоены навыки генерации последовательностей, поиска экстремумов и сортировки массивов. Цель работы достигнута.

\end{document}
