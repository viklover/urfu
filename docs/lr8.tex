\documentclass[12pt,a4paper]{extarticle}
\usepackage{setspace}
\usepackage{indentfirst}
\usepackage{fancyhdr}
\usepackage{lastpage}
\usepackage{fontspec}
\usepackage{geometry}

% Шрифт
\setmainfont{Times New Roman}[
    Path = ./fonts/,
    UprightFont = times.ttf,
    BoldFont = timesbd.ttf,
    ItalicFont = timesi.ttf,
    BoldItalicFont = timesbi.ttf
]

% Поля документа
\geometry{left=2cm}% левое поле
\geometry{right=1.5cm}% правое поле
\geometry{top=1cm}% верхнее поле
\geometry{bottom=2cm}% нижнее поле

\newcommand{\imgh}[3] {
    \begin{figure}[h]
    \center{\includegraphics[width=#1]{#2}}
    \caption{#3}
    \label{ris:#2}
    \end{figure}
}

% Полуторный межстрочный
\onehalfspacing

% Красная строка
\setlength{\parindent}{1.25cm}

\newcommand{\maketitlepage}[1]{
\begin{titlepage}
\thispagestyle{empty}

% Верхняя часть - по центру
\begin{center}
\LARGE Уральский федеральный университет
\end{center}

% Пропускаем пространство до середины
\vspace*{\stretch{2}}

% Средняя часть - по центру
\begin{center}
\bfseries\Huge Отчет по \\
\bfseries\Huge Лабораторной работе №#1
\end{center}

% Пропускаем немного
\vspace*{\stretch{3}}

% Информация о студенте - выравнивание по левому краю
\begin{flushleft}
\large
\textbf{Студент:} Воробьев Михаил Сергеевич, \\
\textbf{Группа:} РИЗ-150916у
\end{flushleft}

% Пропускаем до низа
\vfill

% Нижняя часть - по центру
\begin{center}
\large Екатеринбург, \\
\large 2026 год
\end{center}

\end{titlepage}
}
\usepackage{listings}
\usepackage{xcolor}

\lstset{
    basicstyle=\ttfamily\small,
    keywordstyle=\color{blue},
    commentstyle=\color{gray},
    stringstyle=\color{red},
    frame=single,
    breaklines=true,
    showstringspaces=false
}

\begin{document}

\maketitlepage{8}

% Оглавление
\tableofcontents
\newpage

% Разделы
\section{Цель работы}
\normalfont Получение навыков работы с каталогами и файлами операционной системы, а также с классами ввода/вывода, получение навыков ввода/вывода данных файла через символьные потоки.

\section{Описание задачи}

\begin{enumerate}
    \item Выявить различие в работе программ в примерах 7 и 8
    \item Создать проект, позволяющий из одного, предварительно созданного программными средствами файла, переписать данные, соответствующие условию - в исходном файле содержится две строки в формате UTF-8 и 5 чисел типа double. В результирующий файл переписать вторую строку и положительные числа
    \item Создать проект, позволяющий из одного текстового файла, содержащего несколько строк (тип String) заранее подготовленного текста на русском языке (Пушкин, Лермонтов или другой российский классик на Ваш вкус), построчно переписать в другой текстовый файл слова начинающиеся с согласных букв
\end{enumerate}

\section{Ход выполнения}

\subsection{Задание 1: Сравнение работы программ в примерах 7 и 8}

\textbf{Example7:} Чтение файла построчно с помощью \texttt{BufferedReader} и запись в другой файл через \texttt{BufferedWriter}. Используется буферизация для повышения производительности, выводятся строки в консоль. Обработка исключений через \texttt{try-catch-finally}, в блоке \texttt{finally} закрываются ресурсы и выполняется \texttt{flush}.

\textbf{Example8:} Чтение символов по одному через \texttt{Reader} из различных источников: файл, URL (эмулируется локальным файлом), массив байт. Кодировка задаётся явно (UTF-8). Вывод в консоль с обозначением конца потока (\texttt{"end"}). Буферизация не используется, ресурсы закрываются вручную после каждого блока.

\textbf{Ключевые отличия:} 
\begin{itemize}
    \item Example7 — построчное чтение и запись с буферизацией, Example8 — посимвольное чтение без буфера.
    \item Example7 работает только с файлами, Example8 — с разными источниками данных.
    \item Example7 использует кодировку по умолчанию, Example8 задаёт UTF-8 явно.
\end{itemize}

\subsection{Задание 2}

\begin{lstlisting}[language=Java]
package lr8;

import java.io.*;

public class Task2 {
    public static void main(String[] args) {
        String inputFile = "input_task2.dat";
        String outputFile = "output_task2.dat";

        // Create input file with test data
        try (DataOutputStream dos = new DataOutputStream(
                new FileOutputStream(inputFile))) {
            dos.writeUTF("First line of text");
            dos.writeUTF("Second line of text");
            dos.writeDouble(3.14);
            dos.writeDouble(-2.5);
            dos.writeDouble(7.8);
            dos.writeDouble(-1.0);
            dos.writeDouble(9.99);
            System.out.println("Input file created successfully");
        } catch (IOException e) {
            System.out.println("Error creating input file: " + e);
            return;
        }

        // Read from input and write filtered data to output
        try (DataInputStream dis = new DataInputStream(
                new FileInputStream(inputFile));
             DataOutputStream dos = new DataOutputStream(
                new FileOutputStream(outputFile))) {

            // Read two strings
            String line1 = dis.readUTF();
            String line2 = dis.readUTF();
            System.out.println("Line 1: " + line1);
            System.out.println("Line 2: " + line2);

            // Write second line to output
            dos.writeUTF(line2);

            // Read 5 doubles and write positive ones
            System.out.println("Numbers:");
            for (int i = 0; i < 5; i++) {
                double num = dis.readDouble();
                System.out.println("  " + num);
                if (num > 0) {
                    dos.writeDouble(num);
                }
            }
            System.out.println("Output file created successfully");

        } catch (IOException e) {
            System.out.println("Error: " + e);
        }

        // Verify output file
        System.out.println("\nVerifying output file:");
        try (DataInputStream dis = new DataInputStream(
                new FileInputStream(outputFile))) {
            System.out.println("Second line: " + dis.readUTF());
            System.out.println("Positive numbers:");
            while (dis.available() > 0) {
                System.out.println("  " + dis.readDouble());
            }
        } catch (IOException e) {
            System.out.println("Error reading output: " + e);
        }
    }
}
\end{lstlisting}

\subsection{Задание 3}

\begin{lstlisting}[language=Java]
package lr8;

import java.io.*;
import java.nio.charset.StandardCharsets;

public class Task3 {
    public static void main(String[] args) {
        String inputFile = "input_task3.txt";
        String outputFile = "output_task3.txt";
        String vowels = "аеёиоуыэюяАЕЁИОУЫЭЮЯ";

        // Create input file with Russian text (Pushkin)
        try (BufferedWriter bw = new BufferedWriter(
                new OutputStreamWriter(
                    new FileOutputStream(inputFile), StandardCharsets.UTF_8))) {
            bw.write("Я помню чудное мгновенье"); bw.newLine();
            bw.write("Передо мной явилась ты"); bw.newLine();
            bw.write("Как мимолетное виденье"); bw.newLine();
            bw.write("Как гений чистой красоты"); bw.newLine();
            System.out.println("Input file created with Pushkin's poem");
        } catch (IOException e) {
            System.out.println("Error creating input file: " + e);
            return;
        }

        // Process file
        try (BufferedReader br = new BufferedReader(
                new InputStreamReader(
                    new FileInputStream(inputFile), StandardCharsets.UTF_8));
             BufferedWriter bw = new BufferedWriter(
                new OutputStreamWriter(
                    new FileOutputStream(outputFile), StandardCharsets.UTF_8))) {

            String line;
            int lineNumber = 1;

            while ((line = br.readLine()) != null) {
                String[] words = line.split("\\s+");
                StringBuilder consonantWords = new StringBuilder();
                int count = 0;

                for (String word : words) {
                    if (word.isEmpty()) continue;
                    char firstChar = word.charAt(0);
                    // Check if first char is a letter and not a vowel
                    if (Character.isLetter(firstChar) &&
                        vowels.indexOf(firstChar) == -1) {
                        if (consonantWords.length() > 0) {
                            consonantWords.append(" ");
                        }
                        consonantWords.append(word);
                        count++;
                    }
                }

                bw.write("Line " + lineNumber + " (" + count + " words): ");
                bw.write(consonantWords.toString());
                bw.newLine();
                lineNumber++;
            }
            System.out.println("Output file created successfully");

        } catch (IOException e) {
            System.out.println("Error: " + e);
        }

        // Display output file
        System.out.println("\nOutput file contents:");
        try (BufferedReader br = new BufferedReader(
                new InputStreamReader(
                    new FileInputStream(outputFile), StandardCharsets.UTF_8))) {
            String line;
            while ((line = br.readLine()) != null) {
                System.out.println(line);
            }
        } catch (IOException e) {
            System.out.println("Error reading output: " + e);
        }
    }
}
\end{lstlisting}

\section{Ссылка на GitHub-репозиторий}

Исходный код лабораторной работы доступен по ссылке:

\url{https://github.com/viklover/urfu}

Файлы классов расположены в директории \texttt{src/lr8/}.

\section{Вывод}

В ходе выполнения лабораторной работы были получены практические навыки работы с каталогами и файлами операционной системы, а также с классами ввода/вывода в языке Java. Были изучены и применены различные способы чтения и записи данных через символьные потоки: построчное чтение с использованием \texttt{BufferedReader} и запись через \texttt{BufferedWriter}, а также посимвольное чтение с помощью \texttt{Reader} из различных источников. Работа с буферизацией, обработкой исключений и явным указанием кодировки позволила закрепить навыки эффективного и корректного ввода/вывода данных из файлов.

\end{document}
