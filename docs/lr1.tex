\documentclass[12pt,a4paper]{extarticle}
\usepackage{setspace}
\usepackage{indentfirst}
\usepackage{fancyhdr}
\usepackage{lastpage}
\usepackage{fontspec}
\usepackage{geometry}

% Шрифт
\setmainfont{Times New Roman}[
    Path = ./fonts/,
    UprightFont = times.ttf,
    BoldFont = timesbd.ttf,
    ItalicFont = timesi.ttf,
    BoldItalicFont = timesbi.ttf
]

% Поля документа
\geometry{left=2cm}% левое поле
\geometry{right=1.5cm}% правое поле
\geometry{top=1cm}% верхнее поле
\geometry{bottom=2cm}% нижнее поле

\newcommand{\imgh}[3] {
    \begin{figure}[h]
    \center{\includegraphics[width=#1]{#2}}
    \caption{#3}
    \label{ris:#2}
    \end{figure}
}

% Полуторный межстрочный
\onehalfspacing

% Красная строка
\setlength{\parindent}{1.25cm}

\newcommand{\maketitlepage}[1]{
\begin{titlepage}
\thispagestyle{empty}

% Верхняя часть - по центру
\begin{center}
\LARGE Уральский федеральный университет
\end{center}

% Пропускаем пространство до середины
\vspace*{\stretch{2}}

% Средняя часть - по центру
\begin{center}
\bfseries\Huge Отчет по \\
\bfseries\Huge Лабораторной работе №#1
\end{center}

% Пропускаем немного
\vspace*{\stretch{3}}

% Информация о студенте - выравнивание по левому краю
\begin{flushleft}
\large
\textbf{Студент:} Воробьев Михаил Сергеевич, \\
\textbf{Группа:} РИЗ-150916у
\end{flushleft}

% Пропускаем до низа
\vfill

% Нижняя часть - по центру
\begin{center}
\large Екатеринбург, \\
\large 2026 год
\end{center}

\end{titlepage}
}
\usepackage{listings}
\usepackage{xcolor}

\lstset{
    basicstyle=\ttfamily\small,
    keywordstyle=\color{blue},
    commentstyle=\color{gray},
    stringstyle=\color{red},
    frame=single,
    breaklines=true,
    showstringspaces=false
}

\begin{document}

\maketitlepage{1}

% Оглавление
\tableofcontents
\newpage

% Разделы
\section{Цель работы}
\normalfont Освоить основы программирования на языке Java: работу с консольным вводом/выводом, использование класса Scanner, операции с переменными различных типов (String, int) и базовые арифметические вычисления.

\section{Описание задачи}

Реализовать 10 отдельных Java-классов в пакете lr1. Каждый класс должен содержать метод main, осуществлять ввод данных через Scanner и выводить результат в консоль.

\begin{enumerate}
    \item Пользователь вводит фамилию, имя и отчество — вывести «Hello <ФИО>»
    \item Пользователь вводит имя и возраст — вывести информацию о пользователе
    \item Пользователь вводит день недели, месяц и число — вывести полную дату
    \item Пользователь вводит название месяца и количество дней — вывести сообщение о количестве дней в месяце
    \item По году рождения определить возраст пользователя
    \item Пользователь вводит имя и год рождения — вывести имя и возраст
    \item По возрасту определить год рождения
    \item Пользователь вводит два числа — вывести их сумму
    \item Пользователь вводит число — вывести число на 1 меньше, введённое число, число на 1 больше и квадрат суммы этих трёх чисел
    \item Пользователь вводит два числа — вывести сумму и разность
\end{enumerate}

\section{Ход выполнения}

\subsection{Задание 1: Приветствие по ФИО}

Программа запрашивает у пользователя фамилию, имя и отчество, а затем выводит приветствие. Для ввода строк используется метод nextLine() класса Scanner.

\begin{lstlisting}[language=Java]
package lr1;

import java.util.Scanner;

public class Task1 {
    public static void main(String[] args) {
        Scanner in = new Scanner(System.in);
        System.out.println("Введите фамилию:");
        String lastname = in.nextLine();
        System.out.println("Введите имя:");
        String firstname = in.nextLine();
        System.out.println("Введите отчество:");
        String patronymic = in.nextLine();
        System.out.println("Hello " + lastname + " "
            + firstname + " " + patronymic);
        in.close();
    }
}
\end{lstlisting}

\subsection{Задание 2: Информация о пользователе}

Программа принимает имя (строка) и возраст (целое число), затем выводит эту информацию. Демонстрируется использование nextLine() для строк и nextInt() для чисел.

\begin{lstlisting}[language=Java]
package lr1;

import java.util.Scanner;

public class Task2 {
    public static void main(String[] args) {
        Scanner in = new Scanner(System.in);
        System.out.println("Введите имя:");
        String name = in.nextLine();
        System.out.println("Введите возраст:");
        int age = in.nextInt();
        System.out.println("Имя: " + name);
        System.out.println("Возраст: " + age);
        in.close();
    }
}
\end{lstlisting}

\subsection{Задание 3: Вывод полной даты}

Программа собирает дату из трёх компонентов: день недели, месяц и число. Результат выводится в формате «день недели, число месяц».

\begin{lstlisting}[language=Java]
package lr1;

import java.util.Scanner;

public class Task3 {
    public static void main(String[] args) {
        Scanner in = new Scanner(System.in);
        System.out.println("Введите день недели:");
        String dayOfWeek = in.nextLine();
        System.out.println("Введите месяц:");
        String month = in.nextLine();
        System.out.println("Введите число:");
        int day = in.nextInt();
        System.out.println("Полная дата: " + dayOfWeek
            + ", " + day + " " + month);
        in.close();
    }
}
\end{lstlisting}

\subsection{Задание 4: Количество дней в месяце}

Программа принимает название месяца и количество дней, затем формирует информационное сообщение.

\begin{lstlisting}[language=Java]
package lr1;

import java.util.Scanner;

public class Task4 {
    public static void main(String[] args) {
        Scanner in = new Scanner(System.in);
        System.out.println("Введите название месяца:");
        String month = in.nextLine();
        System.out.println("Введите количество дней:");
        int days = in.nextInt();
        System.out.println("В месяце " + month
            + " " + days + " дней");
        in.close();
    }
}
\end{lstlisting}

\subsection{Задание 5: Вычисление возраста по году рождения}

Программа вычисляет возраст пользователя, вычитая год рождения из текущего года.

\begin{lstlisting}[language=Java]
package lr1;

import java.util.Scanner;

public class Task5 {
    public static void main(String[] args) {
        Scanner in = new Scanner(System.in);
        System.out.println("Введите год рождения:");
        int birthYear = in.nextInt();
        int currentYear = 2026;
        int age = currentYear - birthYear;
        System.out.println("Ваш возраст: " + age);
        in.close();
    }
}
\end{lstlisting}

\subsection{Задание 6: Имя и возраст по году рождения}

Расширение задания 5: программа дополнительно запрашивает имя пользователя и выводит его вместе с вычисленным возрастом.

\begin{lstlisting}[language=Java]
package lr1;

import java.util.Scanner;

public class Task6 {
    public static void main(String[] args) {
        Scanner in = new Scanner(System.in);
        System.out.println("Введите имя:");
        String name = in.nextLine();
        System.out.println("Введите год рождения:");
        int birthYear = in.nextInt();
        int currentYear = 2026;
        int age = currentYear - birthYear;
        System.out.println("Имя: " + name);
        System.out.println("Возраст: " + age);
        in.close();
    }
}
\end{lstlisting}

\subsection{Задание 7: Вычисление года рождения по возрасту}

Обратная задача: по введённому возрасту определяется год рождения.

\begin{lstlisting}[language=Java]
package lr1;

import java.util.Scanner;

public class Task7 {
    public static void main(String[] args) {
        Scanner in = new Scanner(System.in);
        System.out.println("Введите возраст:");
        int age = in.nextInt();
        int currentYear = 2026;
        int birthYear = currentYear - age;
        System.out.println("Год рождения: " + birthYear);
        in.close();
    }
}
\end{lstlisting}

\subsection{Задание 8: Сумма двух чисел}

Простейшая арифметическая операция: сложение двух целых чисел.

\begin{lstlisting}[language=Java]
package lr1;

import java.util.Scanner;

public class Task8 {
    public static void main(String[] args) {
        Scanner in = new Scanner(System.in);
        System.out.println("Введите первое число:");
        int a = in.nextInt();
        System.out.println("Введите второе число:");
        int b = in.nextInt();
        int sum = a + b;
        System.out.println("Сумма: " + sum);
        in.close();
    }
}
\end{lstlisting}

\subsection{Задание 9: Соседние числа и квадрат суммы}

Программа выводит три последовательных числа (введённое и два соседних), а также квадрат их суммы. Демонстрируется работа с несколькими арифметическими операциями.

\begin{lstlisting}[language=Java]
package lr1;

import java.util.Scanner;

public class Task9 {
    public static void main(String[] args) {
        Scanner in = new Scanner(System.in);
        System.out.println("Введите число:");
        int num = in.nextInt();
        int less = num - 1;
        int more = num + 1;
        int sum = less + num + more;
        int square = sum * sum;
        System.out.println("Число на 1 меньше: " + less);
        System.out.println("Введённое число: " + num);
        System.out.println("Число на 1 больше: " + more);
        System.out.println("Квадрат суммы: " + square);
        in.close();
    }
}
\end{lstlisting}

\subsection{Задание 10: Сумма и разность двух чисел}

Программа выполняет две арифметические операции над парой чисел: сложение и вычитание.

\begin{lstlisting}[language=Java]
package lr1;

import java.util.Scanner;

public class Task10 {
    public static void main(String[] args) {
        Scanner in = new Scanner(System.in);
        System.out.println("Введите первое число:");
        int a = in.nextInt();
        System.out.println("Введите второе число:");
        int b = in.nextInt();
        int sum = a + b;
        int diff = a - b;
        System.out.println("Сумма: " + sum);
        System.out.println("Разность: " + diff);
        in.close();
    }
}
\end{lstlisting}

\section{Ссылка на GitHub-репозиторий}

Исходный код лабораторной работы доступен по ссылке:

\url{https://github.com/viklover/urfu}

Файлы классов расположены в директории \texttt{src/lr1/}.

\section{Вывод}

В ходе лабораторной работы были реализованы 10 консольных Java-программ, демонстрирующих базовые возможности языка. Освоены навыки работы с классом Scanner для ввода данных, конкатенации строк, объявления переменных различных типов и выполнения арифметических операций. Цель работы достигнута.

\end{document}
