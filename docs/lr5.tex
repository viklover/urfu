\documentclass[12pt,a4paper]{extarticle}
\usepackage{setspace}
\usepackage{indentfirst}
\usepackage{fancyhdr}
\usepackage{lastpage}
\usepackage{fontspec}
\usepackage{geometry}

% Шрифт
\setmainfont{Times New Roman}[
    Path = ./fonts/,
    UprightFont = times.ttf,
    BoldFont = timesbd.ttf,
    ItalicFont = timesi.ttf,
    BoldItalicFont = timesbi.ttf
]

% Поля документа
\geometry{left=2cm}% левое поле
\geometry{right=1.5cm}% правое поле
\geometry{top=1cm}% верхнее поле
\geometry{bottom=2cm}% нижнее поле

\newcommand{\imgh}[3] {
    \begin{figure}[h]
    \center{\includegraphics[width=#1]{#2}}
    \caption{#3}
    \label{ris:#2}
    \end{figure}
}

% Полуторный межстрочный
\onehalfspacing

% Красная строка
\setlength{\parindent}{1.25cm}

\newcommand{\maketitlepage}[1]{
\begin{titlepage}
\thispagestyle{empty}

% Верхняя часть - по центру
\begin{center}
\LARGE Уральский федеральный университет
\end{center}

% Пропускаем пространство до середины
\vspace*{\stretch{2}}

% Средняя часть - по центру
\begin{center}
\bfseries\Huge Отчет по \\
\bfseries\Huge Лабораторной работе №#1
\end{center}

% Пропускаем немного
\vspace*{\stretch{3}}

% Информация о студенте - выравнивание по левому краю
\begin{flushleft}
\large
\textbf{Студент:} Воробьев Михаил Сергеевич, \\
\textbf{Группа:} РИЗ-150916у
\end{flushleft}

% Пропускаем до низа
\vfill

% Нижняя часть - по центру
\begin{center}
\large Екатеринбург, \\
\large 2026 год
\end{center}

\end{titlepage}
}
\usepackage{listings}
\usepackage{xcolor}

\lstset{
    basicstyle=\ttfamily\small,
    keywordstyle=\color{blue},
    commentstyle=\color{gray},
    stringstyle=\color{red},
    frame=single,
    breaklines=true,
    showstringspaces=false
}

\begin{document}

\maketitlepage{5}

% Оглавление
\tableofcontents
\newpage

% Разделы
\section{Цель работы}
\normalfont Освоить основы объектно-ориентированного программирования в Java: создание классов, работу с полями и методами, инкапсуляцию, перегрузку конструкторов и методов.

\section{Описание задачи}

Реализовать 6 программ в пакете lr5, демонстрирующих создание пользовательских классов с различной функциональностью.

\begin{enumerate}
    \item Класс с закрытым символьным полем и методами доступа
    \item Класс для вывода диапазона символов
    \item Класс с перегруженными конструкторами
    \item Класс с конструктором, разбирающим число типа double
    \item Класс с ограничением значения поля
    \item Класс с полями min и max
\end{enumerate}

\section{Ход выполнения}

\subsection{Задание 1: Класс с символьным полем}

Класс CharHolder содержит закрытое символьное поле и три метода: setValue для присваивания значения, getCode для получения кода символа и print для вывода символа и его кода.

\begin{lstlisting}[language=Java]
package lr5;

public class Task1 {
    static class CharHolder {
        private char symbol;

        public void setValue(char c) {
            symbol = c;
        }

        public int getCode() {
            return (int) symbol;
        }

        public void print() {
            System.out.println("Symbol: " + symbol
                + ", Code: " + (int) symbol);
        }
    }

    public static void main(String[] args) {
        CharHolder holder = new CharHolder();
        holder.setValue('A');
        System.out.println("Code of symbol: " + holder.getCode());
        holder.print();
        holder.setValue('Z');
        holder.print();
    }
}
\end{lstlisting}

\subsection{Задание 2: Диапазон символов}

Класс CharRange содержит два символьных поля и метод printRange, который выводит все символы между значениями полей включительно.

\begin{lstlisting}[language=Java]
package lr5;

public class Task2 {
    static class CharRange {
        public char start;
        public char end;

        public void printRange() {
            char from = (start < end) ? start : end;
            char to = (start < end) ? end : start;
            System.out.print("Symbols from '" + from
                + "' to '" + to + "': ");
            for (char c = from; c <= to; c++) {
                System.out.print(c + " ");
            }
            System.out.println();
        }
    }

    public static void main(String[] args) {
        CharRange range = new CharRange();
        range.start = 'A';
        range.end = 'D';
        range.printRange();
        range.start = 'Z';
        range.end = 'T';
        range.printRange();
    }
}
\end{lstlisting}

\subsection{Задание 3: Перегрузка конструкторов}

Класс TwoIntegers имеет два целочисленных поля и три конструктора: без аргументов, с одним аргументом и с двумя аргументами.

\begin{lstlisting}[language=Java]
package lr5;

public class Task3 {
    static class TwoIntegers {
        public int a;
        public int b;

        public TwoIntegers() {
            a = 0;
            b = 0;
        }

        public TwoIntegers(int value) {
            a = value;
            b = value;
        }

        public TwoIntegers(int first, int second) {
            a = first;
            b = second;
        }

        public void print() {
            System.out.println("a = " + a + ", b = " + b);
        }
    }

    public static void main(String[] args) {
        TwoIntegers obj1 = new TwoIntegers();
        System.out.print("No arguments: ");
        obj1.print();
        TwoIntegers obj2 = new TwoIntegers(5);
        System.out.print("One argument (5): ");
        obj2.print();
        TwoIntegers obj3 = new TwoIntegers(10, 20);
        System.out.print("Two arguments (10, 20): ");
        obj3.print();
    }
}
\end{lstlisting}

\subsection{Задание 4: Разбор числа double}

Класс CharAndInt имеет символьное и целочисленное поле. Конструктор с аргументом double извлекает целую часть как код символа, а дробную часть (десятые и сотые) как целочисленное значение.

\begin{lstlisting}[language=Java]
package lr5;

public class Task4 {
    static class CharAndInt {
        public char symbol;
        public int number;

        public CharAndInt(int code, char c) {
            symbol = c;
            number = code;
        }

        public CharAndInt(double value) {
            int intPart = (int) value;
            symbol = (char) intPart;
            double fractional = value - intPart;
            number = (int) (fractional * 100);
        }

        public void print() {
            System.out.println("Symbol: " + symbol
                + ", Number: " + number);
        }
    }

    public static void main(String[] args) {
        CharAndInt obj1 = new CharAndInt(100, 'X');
        System.out.print("Constructor (int, char): ");
        obj1.print();
        CharAndInt obj2 = new CharAndInt(65.1267);
        System.out.print("Constructor (65.1267): ");
        obj2.print();
        CharAndInt obj3 = new CharAndInt(66.9915);
        System.out.print("Constructor (66.9915): ");
        obj3.print();
    }
}
\end{lstlisting}

\subsection{Задание 5: Ограничение значения поля}

Класс LimitedInt содержит закрытое целочисленное поле. Методы setValue перегружены: без аргументов присваивает 0, с аргументом — ограничивает значение сверху числом 100.

\begin{lstlisting}[language=Java]
package lr5;

public class Task5 {
    static class LimitedInt {
        private int value;

        public LimitedInt() {
            value = 0;
        }

        public LimitedInt(int val) {
            setValue(val);
        }

        public void setValue() {
            value = 0;
        }

        public void setValue(int val) {
            if (val > 100) {
                value = 100;
            } else {
                value = val;
            }
        }

        public int getValue() {
            return value;
        }
    }

    public static void main(String[] args) {
        LimitedInt obj1 = new LimitedInt();
        System.out.println("Default constructor: " + obj1.getValue());
        LimitedInt obj2 = new LimitedInt(50);
        System.out.println("Constructor with 50: " + obj2.getValue());
        LimitedInt obj3 = new LimitedInt(150);
        System.out.println("Constructor with 150: " + obj3.getValue());
        obj1.setValue(75);
        System.out.println("After setValue(75): " + obj1.getValue());
        obj1.setValue(200);
        System.out.println("After setValue(200): " + obj1.getValue());
        obj1.setValue();
        System.out.println("After setValue(): " + obj1.getValue());
    }
}
\end{lstlisting}

\subsection{Задание 6: Поля min и max}

Класс MinMax содержит два закрытых поля min и max. Метод setValues обновляет поля так, чтобы min всегда содержало минимальное, а max — максимальное из всех переданных значений.

\begin{lstlisting}[language=Java]
package lr5;

public class Task6 {
    static class MinMax {
        private int min;
        private int max;

        public MinMax() {
            min = 0;
            max = 0;
        }

        public MinMax(int a) {
            min = a;
            max = a;
        }

        public MinMax(int a, int b) {
            setValues(a, b);
        }

        public void setValues(int a) {
            updateMinMax(a);
        }

        public void setValues(int a, int b) {
            updateMinMax(a);
            updateMinMax(b);
        }

        private void updateMinMax(int value) {
            if (value < min) {
                min = value;
            }
            if (value > max) {
                max = value;
            }
        }

        public void print() {
            System.out.println("min = " + min + ", max = " + max);
        }
    }

    public static void main(String[] args) {
        MinMax obj1 = new MinMax(5, 10);
        System.out.print("Constructor (5, 10): ");
        obj1.print();
        obj1.setValues(3);
        System.out.print("After setValues(3): ");
        obj1.print();
        obj1.setValues(15, 1);
        System.out.print("After setValues(15, 1): ");
        obj1.print();
        MinMax obj2 = new MinMax(50);
        System.out.print("Constructor (50): ");
        obj2.print();
    }
}
\end{lstlisting}

\section{Ссылка на GitHub-репозиторий}

Исходный код лабораторной работы доступен по ссылке:

\url{https://github.com/viklover/urfu}

Файлы классов расположены в директории \texttt{src/lr5/}.

\section{Вывод}

В ходе лабораторной работы были реализованы программы, демонстрирующие основы ООП в Java. Освоены навыки создания классов с полями и методами, инкапсуляции данных с помощью модификатора private, перегрузки конструкторов и методов. Цель работы достигнута.

\end{document}
