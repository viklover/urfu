\documentclass[12pt,a4paper]{extarticle}
\usepackage{setspace}
\usepackage{indentfirst}
\usepackage{fancyhdr}
\usepackage{lastpage}
\usepackage{fontspec}
\usepackage{geometry}

% Шрифт
\setmainfont{Times New Roman}[
    Path = ./fonts/,
    UprightFont = times.ttf,
    BoldFont = timesbd.ttf,
    ItalicFont = timesi.ttf,
    BoldItalicFont = timesbi.ttf
]

% Поля документа
\geometry{left=2cm}% левое поле
\geometry{right=1.5cm}% правое поле
\geometry{top=1cm}% верхнее поле
\geometry{bottom=2cm}% нижнее поле

\newcommand{\imgh}[3] {
    \begin{figure}[h]
    \center{\includegraphics[width=#1]{#2}}
    \caption{#3}
    \label{ris:#2}
    \end{figure}
}

% Полуторный межстрочный
\onehalfspacing

% Красная строка
\setlength{\parindent}{1.25cm}

\newcommand{\maketitlepage}[1]{
\begin{titlepage}
\thispagestyle{empty}

% Верхняя часть - по центру
\begin{center}
\LARGE Уральский федеральный университет
\end{center}

% Пропускаем пространство до середины
\vspace*{\stretch{2}}

% Средняя часть - по центру
\begin{center}
\bfseries\Huge Отчет по \\
\bfseries\Huge Лабораторной работе №#1
\end{center}

% Пропускаем немного
\vspace*{\stretch{3}}

% Информация о студенте - выравнивание по левому краю
\begin{flushleft}
\large
\textbf{Студент:} Воробьев Михаил Сергеевич, \\
\textbf{Группа:} РИЗ-150916у
\end{flushleft}

% Пропускаем до низа
\vfill

% Нижняя часть - по центру
\begin{center}
\large Екатеринбург, \\
\large 2026 год
\end{center}

\end{titlepage}
}
\usepackage{listings}
\usepackage{xcolor}

\lstset{
    basicstyle=\ttfamily\small,
    keywordstyle=\color{blue},
    commentstyle=\color{gray},
    stringstyle=\color{red},
    frame=single,
    breaklines=true,
    showstringspaces=false
}

\begin{document}

\maketitlepage{7}

% Оглавление
\tableofcontents
\newpage

% Разделы
\section{Цель работы}
\normalfont Освоить механизм наследования в Java: создание иерархий классов, переопределение и перегрузку методов, полиморфизм, конструкторы копирования.

\section{Описание задачи}

Реализовать 5 программ в пакете lr7, демонстрирующих различные аспекты наследования классов.

\begin{enumerate}
    \item Суперкласс и подкласс с переопределением toString()
    \item Наследование с перегрузкой методов присваивания
    \item Цепочка наследования из трёх классов с перегрузкой методов
    \item Цепочка наследования с конструкторами копирования
    \item Два подкласса от одного суперкласса, демонстрация полиморфизма
\end{enumerate}

\section{Ход выполнения}

\subsection{Задание 1: Переопределение toString()}

Суперкласс содержит приватное текстовое поле и переопределённый метод toString(). Подкласс добавляет ещё одно текстовое поле и также переопределяет toString().

\begin{lstlisting}[language=Java]
package lr7;

public class Task1 {
    static class SuperClass {
        private String text;

        public SuperClass(String text) {
            this.text = text;
        }

        @Override
        public String toString() {
            return "SuperClass[text=" + text + "]";
        }

        protected String getText() {
            return text;
        }
    }

    static class SubClass extends SuperClass {
        private String additionalText;

        public SubClass(String text) {
            super(text);
            this.additionalText = "";
        }

        public SubClass(String text, String additionalText) {
            super(text);
            this.additionalText = additionalText;
        }

        @Override
        public String toString() {
            return "SubClass[text=" + getText()
                + ", additionalText=" + additionalText + "]";
        }
    }

    public static void main(String[] args) {
        SuperClass obj1 = new SuperClass("Hello");
        System.out.println(obj1.toString());
        SubClass obj2 = new SubClass("World");
        System.out.println(obj2.toString());
        SubClass obj3 = new SubClass("Java", "Programming");
        System.out.println(obj3.toString());
    }
}
\end{lstlisting}

\subsection{Задание 2: Перегрузка методов присваивания}

Суперкласс содержит метод setValue и метод getLength. Подкласс перегружает setValue для работы с разными комбинациями параметров.

\begin{lstlisting}[language=Java]
package lr7;

public class Task2 {
    static class SuperClass {
        private String text;

        public SuperClass(String text) {
            this.text = text;
        }

        public void setValue(String text) {
            this.text = text;
        }

        public int getLength() {
            return text.length();
        }

        protected String getText() {
            return text;
        }
    }

    static class SubClass extends SuperClass {
        public int number;

        public SubClass(String text, int number) {
            super(text);
            this.number = number;
        }

        public void setValue() {
            super.setValue("");
            this.number = 0;
        }

        @Override
        public void setValue(String text) {
            super.setValue(text);
        }

        public void setValue(int number) {
            this.number = number;
        }

        public void setValue(String text, int number) {
            super.setValue(text);
            this.number = number;
        }
    }

    public static void main(String[] args) {
        SubClass obj = new SubClass("Hello", 42);
        obj.setValue("Java", 100);
        obj.setValue(999);
        obj.setValue();
    }
}
\end{lstlisting}

\subsection{Задание 3: Цепочка из трёх классов}

Три класса образуют цепочку наследования. Каждый класс добавляет своё поле и перегружает метод setValue с увеличивающимся числом параметров.

\begin{lstlisting}[language=Java]
package lr7;

public class Task3 {
    static class FirstClass {
        public int number;

        public FirstClass(int number) {
            this.number = number;
        }

        public void setValue(int number) {
            this.number = number;
        }

        @Override
        public String toString() {
            return "FirstClass[number=" + number + "]";
        }
    }

    static class SecondClass extends FirstClass {
        public char symbol;

        public SecondClass(int number, char symbol) {
            super(number);
            this.symbol = symbol;
        }

        public void setValue(int number, char symbol) {
            this.number = number;
            this.symbol = symbol;
        }

        @Override
        public String toString() {
            return "SecondClass[number=" + number
                + ", symbol=" + symbol + "]";
        }
    }

    static class ThirdClass extends SecondClass {
        public String text;

        public ThirdClass(int number, char symbol, String text) {
            super(number, symbol);
            this.text = text;
        }

        public void setValue(int number, char symbol, String text) {
            this.number = number;
            this.symbol = symbol;
            this.text = text;
        }

        @Override
        public String toString() {
            return "ThirdClass[number=" + number + ", symbol="
                + symbol + ", text=" + text + "]";
        }
    }

    public static void main(String[] args) {
        FirstClass obj1 = new FirstClass(10);
        System.out.println(obj1.toString());
        SecondClass obj2 = new SecondClass(20, 'A');
        System.out.println(obj2.toString());
        ThirdClass obj3 = new ThirdClass(30, 'B', "Hello");
        System.out.println(obj3.toString());
    }
}
\end{lstlisting}

\subsection{Задание 4: Конструкторы копирования}

Цепочка из трёх классов, каждый из которых имеет конструктор для создания объекта по значениям полей и конструктор копирования.

\begin{lstlisting}[language=Java]
package lr7;

public class Task4 {
    static class FirstClass {
        public char symbol;

        public FirstClass(char symbol) {
            this.symbol = symbol;
        }

        public FirstClass(FirstClass other) {
            this.symbol = other.symbol;
        }
    }

    static class SecondClass extends FirstClass {
        public String text;

        public SecondClass(char symbol, String text) {
            super(symbol);
            this.text = text;
        }

        public SecondClass(SecondClass other) {
            super(other.symbol);
            this.text = other.text;
        }
    }

    static class ThirdClass extends SecondClass {
        public int number;

        public ThirdClass(char symbol, String text, int number) {
            super(symbol, text);
            this.number = number;
        }

        public ThirdClass(ThirdClass other) {
            super(other.symbol, other.text);
            this.number = other.number;
        }
    }

    public static void main(String[] args) {
        ThirdClass obj = new ThirdClass('A', "Hello", 42);
        System.out.println("Original: " + obj.symbol
            + ", " + obj.text + ", " + obj.number);
        ThirdClass copy = new ThirdClass(obj);
        System.out.println("Copy: " + copy.symbol
            + ", " + copy.text + ", " + copy.number);
    }
}
\end{lstlisting}

\subsection{Задание 5: Полиморфизм}

Суперкласс и два подкласса, каждый переопределяет метод display(). Демонстрируется полиморфизм через переменную типа суперкласса.

\begin{lstlisting}[language=Java]
package lr7;

public class Task5 {
    static class SuperClass {
        private String text;

        public SuperClass(String text) {
            this.text = text;
        }

        public void display() {
            System.out.println("SuperClass: text=" + text);
        }

        protected String getText() {
            return text;
        }
    }

    static class SubClassA extends SuperClass {
        protected int number;

        public SubClassA(String text, int number) {
            super(text);
            this.number = number;
        }

        @Override
        public void display() {
            System.out.println("SubClassA: text=" + getText()
                + ", number=" + number);
        }
    }

    static class SubClassB extends SuperClass {
        protected char symbol;

        public SubClassB(String text, char symbol) {
            super(text);
            this.symbol = symbol;
        }

        @Override
        public void display() {
            System.out.println("SubClassB: text=" + getText()
                + ", symbol=" + symbol);
        }
    }

    public static void main(String[] args) {
        SuperClass obj1 = new SuperClass("Base");
        SubClassA obj2 = new SubClassA("Alpha", 100);
        SubClassB obj3 = new SubClassB("Beta", 'X');

        obj1.display();
        obj2.display();
        obj3.display();

        System.out.println("\nPolymorphism:");
        SuperClass ref;
        ref = obj2;
        ref.display();
        ref = obj3;
        ref.display();
    }
}
\end{lstlisting}

\section{Ссылка на GitHub-репозиторий}

Исходный код лабораторной работы доступен по ссылке:

\url{https://github.com/viklover/urfu}

Файлы классов расположены в директории \texttt{src/lr7/}.

\section{Вывод}

В ходе лабораторной работы были реализованы программы, демонстрирующие механизм наследования в Java. Освоены навыки создания иерархий классов, переопределения методов (override), перегрузки методов (overload), использования конструкторов копирования и полиморфизма. Цель работы достигнута.

\end{document}
