\documentclass[a4paper,12pt]{article}
\usepackage{setspace}
\usepackage{indentfirst}
\usepackage{fancyhdr}
\usepackage{lastpage}
\usepackage{fontspec}
\usepackage{geometry}

% Шрифт
\setmainfont{Times New Roman}[
    Path = ./fonts/,
    UprightFont = times.ttf,
    BoldFont = timesbd.ttf,
    ItalicFont = timesi.ttf,
    BoldItalicFont = timesbi.ttf
]

% Поля документа
\geometry{left=2cm}% левое поле
\geometry{right=1.5cm}% правое поле
\geometry{top=1cm}% верхнее поле
\geometry{bottom=2cm}% нижнее поле

\newcommand{\imgh}[3] {
    \begin{figure}[h]
    \center{\includegraphics[width=#1]{#2}}
    \caption{#3}
    \label{ris:#2}
    \end{figure}
}

% Полуторный межстрочный
\onehalfspacing

% Красная строка
\setlength{\parindent}{1.25cm}

% Колонтитулы
\pagestyle{fancy}
\fancyhf{}
% \fancyhead[C]{\small Заголовок документа}
\fancyfoot[C]{\small Страница \thepage\ из \pageref{LastPage}}

% \graphicspath{{images/}}

% чтобы underline переносился...
\usepackage{setspace}
\usepackage{indentfirst}
\usepackage{fancyhdr}
\usepackage{lastpage}
\usepackage{fontspec}
\usepackage{geometry}

% Шрифт
\setmainfont{Times New Roman}[
    Path = ./fonts/,
    UprightFont = times.ttf,
    BoldFont = timesbd.ttf,
    ItalicFont = timesi.ttf,
    BoldItalicFont = timesbi.ttf
]

% Поля документа
\geometry{left=2cm}% левое поле
\geometry{right=1.5cm}% правое поле
\geometry{top=1cm}% верхнее поле
\geometry{bottom=2cm}% нижнее поле

\newcommand{\imgh}[3] {
    \begin{figure}[h]
    \center{\includegraphics[width=#1]{#2}}
    \caption{#3}
    \label{ris:#2}
    \end{figure}
}

% Полуторный межстрочный
\onehalfspacing

% Красная строка
\setlength{\parindent}{1.25cm}

\newcommand{\maketitlepage}[1]{
\begin{titlepage}
\thispagestyle{empty}

% Верхняя часть - по центру
\begin{center}
\LARGE Уральский федеральный университет
\end{center}

% Пропускаем пространство до середины
\vspace*{\stretch{2}}

% Средняя часть - по центру
\begin{center}
\bfseries\Huge Отчет по \\
\bfseries\Huge Лабораторной работе №#1
\end{center}

% Пропускаем немного
\vspace*{\stretch{3}}

% Информация о студенте - выравнивание по левому краю
\begin{flushleft}
\large
\textbf{Студент:} Воробьев Михаил Сергеевич, \\
\textbf{Группа:} РИЗ-150916у
\end{flushleft}

% Пропускаем до низа
\vfill

% Нижняя часть - по центру
\begin{center}
\large Екатеринбург, \\
\large 2026 год
\end{center}

\end{titlepage}
}
\usepackage[normalem]{ulem}

% переменные title

\newcommand{\titlepageChair}{ИВТиИБ}
\newcommand{\titlepageLabNo}{1}
\newcommand{\titlepageSubject}{Цифровая обработка сигналов}
\newcommand{\titlepageLRArticul}{230100.68.11.000 О}

\newcommand{\titlepageLabTitle}{
  \parbox{14cm}{
    \center
    \uline{Работа в среде Matlab. Структура Signal Processing Toolbox. Генерация сигналов. Свертка.}
  }
}

\newcommand{\titlepageGroup}{8ИВТ-41}
\newcommand{\titlepageStudent}{А.Ю. Смолянинов}
\newcommand{\titlepageTeacherPosition}{доцент, к.т.н.}
\newcommand{\titlepageTeacher}{А.C. Шатохин}

\begin{document}


\newbox{\lbox}
\savebox{\lbox}{\hbox{Пупкин Иван Иванович}}
\newlength{\maxl}
\setlength{\maxl}{\wd\lbox}
\hfill\parbox{11cm} {
    \hspace*{5cm}\hspace*{-5cm}Студент:\hfill\hbox to\maxl{Тест Пользователь\hfill}\\
    \hspace*{5cm}\hspace*{-5cm}Преподаватель:\hfill\hbox to\maxl{Пупкин Иван Иванович}\\
    \\
    \hspace*{5cm}\hspace*{-5cm}Группа:\hfill\hbox to\maxl{NNN}\\
}
\newpage
\section{Постановка задачи}

\begin{center}
    \LARGE\textbf{НАЗВАНИЕ ДОКУМЕНТА}
    
    \vspace{0.5cm}
    
    \large Подзаголовок или описание
    
    \vspace{1cm}
\end{center}

\section{Введение}

Это типичный документ в стиле Microsoft Word. Текст автоматически выравнивается по ширине, есть красные строки, колонтитулы и привычное форматирование.

Основные преимущества такого подхода:
\begin{itemize}
    \item Профессиональный внешний вид
    \item Автоматическая нумерация
    \item Согласованность стиля
    \item Качественная верстка
\end{itemize}

\section{Основная часть}

\subsection{Структура документа}

Документы в LaTeX, как и в Word, имеют иерархическую структуру:

\subsubsection{Подподраздел}

Можно создавать несколько уровней вложенности.

\subsection{Формулы}

Математические формулы выглядят красиво: $E = mc^2$ или:

\[
\int_{a}^{b} f(x) \, dx = F(b) - F(a)
\]

\section{Таблицы}

\begin{center}
\begin{tabular}{|l|c|r|}
    \hline
    \textbf{Заголовок 1} & \textbf{Заголовок 2} & \textbf{Заголовок 3} \\
    \hline
    Левое выравнивание & По центру & Правое выравнивание \\
    \hline
    Данные 1 & 123.45 & 100 \\
    Данные 2 & 67.89 & 200 \\
    \hline
\end{tabular}
\end{center}

\section{Заключение}

Этот документ выглядит почти как созданный в Word, но обладает всеми преимуществами LaTeX: стабильность, контроль версий, математические возможности.

\begin{flushright}
    \textit{Автор документа}
    
    \today
\end{flushright}

\end{document}