\documentclass[a4paper,12pt]{article}
\usepackage[utf8]{inputenc}
\usepackage[T2A]{fontenc}
\usepackage[russian]{babel}
\usepackage{geometry}
\usepackage{setspace}
\usepackage{indentfirst}
\usepackage{fancyhdr}
\usepackage{lastpage}
\usepackage{fontspec} 
\setmainfont[Ligatures={TeX,Historic}]{Times New Roman}

% Поля как в Word
\geometry{
    left=3cm,
    right=2cm,
    top=2cm,
    bottom=2cm,
    headheight=15pt
}

% Полуторный межстрочный
\onehalfspacing

% Красная строка
\setlength{\parindent}{1.25cm}

% Колонтитулы
\pagestyle{fancy}
\fancyhf{}
\fancyhead[C]{\small Заголовок документа}
\fancyfoot[C]{\small Страница \thepage\ из \pageref{LastPage}}

\begin{document}

\begin{center}
    \LARGE\textbf{НАЗВАНИЕ ДОКУМЕНТА}
    
    \vspace{0.5cm}
    
    \large Подзаголовок или описание
    
    \vspace{1cm}
\end{center}

\section{Введение}

Это типичный документ в стиле Microsoft Word. Текст автоматически выравнивается по ширине, есть красные строки, колонтитулы и привычное форматирование.

Основные преимущества такого подхода:
\begin{itemize}
    \item Профессиональный внешний вид
    \item Автоматическая нумерация
    \item Согласованность стиля
    \item Качественная верстка
\end{itemize}

\section{Основная часть}

\subsection{Структура документа}

Документы в LaTeX, как и в Word, имеют иерархическую структуру:

\subsubsection{Подподраздел}

Можно создавать несколько уровней вложенности.

\subsection{Формулы}

Математические формулы выглядят красиво: $E = mc^2$ или:

\[
\int_{a}^{b} f(x) \, dx = F(b) - F(a)
\]

\section{Таблицы}

\begin{center}
\begin{tabular}{|l|c|r|}
    \hline
    \textbf{Заголовок 1} & \textbf{Заголовок 2} & \textbf{Заголовок 3} \\
    \hline
    Левое выравнивание & По центру & Правое выравнивание \\
    \hline
    Данные 1 & 123.45 & 100 \\
    Данные 2 & 67.89 & 200 \\
    \hline
\end{tabular}
\end{center}

\section{Заключение}

Этот документ выглядит почти как созданный в Word, но обладает всеми преимуществами LaTeX: стабильность, контроль версий, математические возможности.

\begin{flushright}
    \textit{Автор документа}
    
    \today
\end{flushright}

\end{document}