\documentclass[12pt,a4paper]{extarticle}
\usepackage{setspace}
\usepackage{indentfirst}
\usepackage{fancyhdr}
\usepackage{lastpage}
\usepackage{fontspec}
\usepackage{geometry}

% Шрифт
\setmainfont{Times New Roman}[
    Path = ./fonts/,
    UprightFont = times.ttf,
    BoldFont = timesbd.ttf,
    ItalicFont = timesi.ttf,
    BoldItalicFont = timesbi.ttf
]

% Поля документа
\geometry{left=2cm}% левое поле
\geometry{right=1.5cm}% правое поле
\geometry{top=1cm}% верхнее поле
\geometry{bottom=2cm}% нижнее поле

\newcommand{\imgh}[3] {
    \begin{figure}[h]
    \center{\includegraphics[width=#1]{#2}}
    \caption{#3}
    \label{ris:#2}
    \end{figure}
}

% Полуторный межстрочный
\onehalfspacing

% Красная строка
\setlength{\parindent}{1.25cm}

\newcommand{\maketitlepage}[1]{
\begin{titlepage}
\thispagestyle{empty}

% Верхняя часть - по центру
\begin{center}
\LARGE Уральский федеральный университет
\end{center}

% Пропускаем пространство до середины
\vspace*{\stretch{2}}

% Средняя часть - по центру
\begin{center}
\bfseries\Huge Отчет по \\
\bfseries\Huge Лабораторной работе №#1
\end{center}

% Пропускаем немного
\vspace*{\stretch{3}}

% Информация о студенте - выравнивание по левому краю
\begin{flushleft}
\large
\textbf{Студент:} Воробьев Михаил Сергеевич, \\
\textbf{Группа:} РИЗ-150916у
\end{flushleft}

% Пропускаем до низа
\vfill

% Нижняя часть - по центру
\begin{center}
\large Екатеринбург, \\
\large 2026 год
\end{center}

\end{titlepage}
}
\usepackage{listings}
\usepackage{xcolor}

\lstset{
    basicstyle=\ttfamily\small,
    keywordstyle=\color{blue},
    commentstyle=\color{gray},
    stringstyle=\color{red},
    frame=single,
    breaklines=true,
    showstringspaces=false
}

\begin{document}

\maketitlepage{4}

% Оглавление
\tableofcontents
\newpage

% Разделы
\section{Цель работы}
\normalfont Освоить работу с двумерными массивами в языке Java: создание, заполнение, вывод, транспонирование и модификацию матриц.

\section{Описание задачи}

Реализовать 7 программ в пакете lr4, демонстрирующих работу с вложенными циклами и двумерными массивами.

\begin{enumerate}
    \item Вывод прямоугольника 23$\times$11 с помощью вложенных циклов
    \item Вывод прямоугольного треугольника
    \item Двумерный массив, заполненный цифрами 2
    \item Двумерный массив в форме прямоугольного треугольника
    \item Транспонирование массива
    \item Удаление случайной строки и столбца из массива
    \item Заполнение массива «змейкой» по периметру
\end{enumerate}

\section{Ход выполнения}

\subsection{Задание 1: Прямоугольник}

Программа выводит прямоугольник размером 23$\times$11 символов с помощью вложенных циклов. Внешний цикл перебирает строки, внутренний — столбцы.

\begin{lstlisting}[language=Java]
package lr4;

public class Task1 {
    public static void main(String[] args) {
        int width = 23;
        int height = 11;
        for (int i = 0; i < height; i++) {
            for (int j = 0; j < width; j++) {
                System.out.print("*");
            }
            System.out.println();
        }
    }
}
\end{lstlisting}

\subsection{Задание 2: Прямоугольный треугольник}

Программа выводит прямоугольный треугольник. В каждой строке количество символов равно номеру строки.

\begin{lstlisting}[language=Java]
package lr4;

public class Task2 {
    public static void main(String[] args) {
        int size = 10;
        for (int i = 1; i <= size; i++) {
            for (int j = 0; j < i; j++) {
                System.out.print("*");
            }
            System.out.println();
        }
    }
}
\end{lstlisting}

\subsection{Задание 3: Массив из двоек}

Программа создаёт двумерный массив, заполняет его цифрами 2 и выводит в виде прямоугольника.

\begin{lstlisting}[language=Java]
package lr4;

public class Task3 {
    public static void main(String[] args) {
        int rows = 5;
        int cols = 8;
        int[][] arr = new int[rows][cols];
        for (int i = 0; i < rows; i++) {
            for (int j = 0; j < cols; j++) {
                arr[i][j] = 2;
            }
        }
        System.out.println("Rectangle of 2s:");
        for (int i = 0; i < rows; i++) {
            for (int j = 0; j < cols; j++) {
                System.out.print(arr[i][j] + " ");
            }
            System.out.println();
        }
    }
}
\end{lstlisting}

\subsection{Задание 4: Треугольный массив чисел}

Программа создаёт ступенчатый (jagged) массив в форме прямоугольного треугольника. Каждая строка имеет длину, равную номеру строки плюс один.

\begin{lstlisting}[language=Java]
package lr4;

public class Task4 {
    public static void main(String[] args) {
        int size = 5;
        int[][] arr = new int[size][];
        int num = 1;
        for (int i = 0; i < size; i++) {
            arr[i] = new int[i + 1];
            for (int j = 0; j <= i; j++) {
                arr[i][j] = num++;
            }
        }
        System.out.println("Right triangle of numbers:");
        for (int i = 0; i < size; i++) {
            for (int j = 0; j < arr[i].length; j++) {
                System.out.print(arr[i][j] + " ");
            }
            System.out.println();
        }
    }
}
\end{lstlisting}

\subsection{Задание 5: Транспонирование массива}

Программа создаёт массив 3$\times$5, заполняет случайными числами и транспонирует его в массив 5$\times$3. При транспонировании элемент [i][j] становится элементом [j][i].

\begin{lstlisting}[language=Java]
package lr4;

import java.util.Random;

public class Task5 {
    public static void main(String[] args) {
        int rows = 3;
        int cols = 5;
        int[][] arr = new int[rows][cols];
        Random rand = new Random();
        for (int i = 0; i < rows; i++) {
            for (int j = 0; j < cols; j++) {
                arr[i][j] = rand.nextInt(10);
            }
        }
        System.out.println("Original array " + rows + "x" + cols + ":");
        for (int i = 0; i < rows; i++) {
            for (int j = 0; j < cols; j++) {
                System.out.print(arr[i][j] + " ");
            }
            System.out.println();
        }
        int[][] transposed = new int[cols][rows];
        for (int i = 0; i < rows; i++) {
            for (int j = 0; j < cols; j++) {
                transposed[j][i] = arr[i][j];
            }
        }
        System.out.println("Transposed array " + cols + "x" + rows + ":");
        for (int i = 0; i < cols; i++) {
            for (int j = 0; j < rows; j++) {
                System.out.print(transposed[i][j] + " ");
            }
            System.out.println();
        }
    }
}
\end{lstlisting}

\subsection{Задание 6: Удаление строки и столбца}

Программа создаёт массив, затем удаляет из него случайную строку и столбец, формируя новый массив меньшего размера.

\begin{lstlisting}[language=Java]
package lr4;

import java.util.Random;

public class Task6 {
    public static void main(String[] args) {
        int rows = 4;
        int cols = 5;
        int[][] arr = new int[rows][cols];
        int num = 1;
        for (int i = 0; i < rows; i++) {
            for (int j = 0; j < cols; j++) {
                arr[i][j] = num++;
            }
        }
        System.out.println("Original array:");
        for (int i = 0; i < rows; i++) {
            for (int j = 0; j < cols; j++) {
                System.out.print(arr[i][j] + "\t");
            }
            System.out.println();
        }
        Random rand = new Random();
        int delRow = rand.nextInt(rows);
        int delCol = rand.nextInt(cols);
        System.out.println("Deleting row " + delRow
            + " and column " + delCol);
        int[][] result = new int[rows - 1][cols - 1];
        int newI = 0;
        for (int i = 0; i < rows; i++) {
            if (i == delRow) continue;
            int newJ = 0;
            for (int j = 0; j < cols; j++) {
                if (j == delCol) continue;
                result[newI][newJ] = arr[i][j];
                newJ++;
            }
            newI++;
        }
        System.out.println("Result array:");
        for (int i = 0; i < rows - 1; i++) {
            for (int j = 0; j < cols - 1; j++) {
                System.out.print(result[i][j] + "\t");
            }
            System.out.println();
        }
    }
}
\end{lstlisting}

\subsection{Задание 7: Заполнение змейкой}

Программа заполняет двумерный массив по спирали (змейкой по периметру): первая строка слева направо, последний столбец сверху вниз, последняя строка справа налево, первый столбец снизу вверх, и так далее к центру.

\begin{lstlisting}[language=Java]
package lr4;

public class Task7 {
    public static void main(String[] args) {
        int rows = 4;
        int cols = 5;
        int[][] arr = new int[rows][cols];
        int num = 1;
        int top = 0, bottom = rows - 1;
        int left = 0, right = cols - 1;
        while (top <= bottom && left <= right) {
            for (int j = left; j <= right; j++) {
                arr[top][j] = num++;
            }
            top++;
            for (int i = top; i <= bottom; i++) {
                arr[i][right] = num++;
            }
            right--;
            if (top <= bottom) {
                for (int j = right; j >= left; j--) {
                    arr[bottom][j] = num++;
                }
                bottom--;
            }
            if (left <= right) {
                for (int i = bottom; i >= top; i--) {
                    arr[i][left] = num++;
                }
                left++;
            }
        }
        System.out.println("Spiral fill:");
        for (int i = 0; i < rows; i++) {
            for (int j = 0; j < cols; j++) {
                System.out.print(arr[i][j] + "\t");
            }
            System.out.println();
        }
    }
}
\end{lstlisting}

\section{Ссылка на GitHub-репозиторий}

Исходный код лабораторной работы доступен по ссылке:

\url{https://github.com/viklover/urfu}

Файлы классов расположены в директории \texttt{src/lr4/}.

\section{Вывод}

В ходе лабораторной работы были реализованы программы, демонстрирующие работу с двумерными массивами. Освоены навыки создания и вывода матриц, работы со ступенчатыми массивами, транспонирования, удаления элементов и заполнения по спирали. Цель работы достигнута.

\end{document}
